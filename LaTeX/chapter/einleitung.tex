\section{Einleitung}
Die Collision Avoidance Assistance ist eine Forschungs- und Entwicklungsinitiative, die darauf abzielt, ein fortschrittliches System zu entwickeln, was die Sicherheit des Fahrers durch Warnungen und autonomes Eingreifen erhöht, um potenzielle Kollisionen zu verhindern. Diese Dokumentation gibt einen Überblick über das Projekt, einschließlich seiner Ziele, Schlüsselkomponenten und Funktionalitäten.

\subsection{Zielsetzung}
Die Hauptziele des Kollisionsvermeidungs-Assistenzprojekts sind wie folgt:
\begin{itemize}
	\item Erhöhung der Sicherheit des Fahrers durch Erkennung und Vermeidung potenzieller Kollisionsszenarien.
	\item Rechtzeitige und genaue Warnungen, um den Fahrer vor möglichen Gefahren zu warnen.
	\item Falls erforderlich, autonom durch Bremsen oder Lenken eingreifen, um Kollisionen zu vermeiden.
	\item Nutzung verschiedener Sensoren, wie Radar, Kamera oder LiDAR, um die Umgebung zu überwachen und potenzielle Kollisionsrisiken zu erkennen.
	\item Entwicklung eines robusten und zuverlässigen Systems, das unter verschiedenen Fahrbedingungen effektiv arbeiten kann.
\end{itemize}

\subsection{System-Vision}
Unsere Vision ist es, ein zuverlässiges Collision Avoidance Assist-System zu entwickeln, das durch Sensoren potenzielle Kollisionen frühzeitig erkennt und den Fahrer alarmiert oder autonom eingreift, um Unfälle zu verhindern. Das Ziel ist es, die Verkehrssicherheit zu erhöhen und das Unfallrisiko zu reduzieren.
